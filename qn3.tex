\section{Question 3}

In this section, question 3 is discussed.

I was unable to complete this question, however I am describing my approach to it below:

\subsection{3 (a)}
> Since the likelihood is to be written for a distribution that has positive, countable, discrete samples, we can write down the maximum likelihood using a poisson distribution.

> The log likelihood can be written as:
 $$ -\ln{L(p)} = -\sum\limits_{i=0}^{N-1} {(y_i \ln{\mu(x_i/\mathbf{p})} - \mu(x_i/\mathbf{p})- \ln{y_i!} )} $$
Where $x_i$ can be gotten from checking the range of the radii in the respective mass-bin files. Corresponding y_i for the model is drawn form random realizations of the distribution n(x) in equation.2, by using random values of a, b, c generated in the specified range in question 2(a). (Values of A at these points are found using the 3D interpolator written for question 2(h)). 
$\mathbf{p}$ is the vector containing the parameters (a,b,c) of the data.

> To find the (a,b,c) that maximize this likelihood, the above log likelihood equation is differentiated and equated to zero. 
 
 \subsection{3 b)}
If the values of (a,b,c) have significant number of outliers, robust fitting would be suitable, since interpolation would wrongly estimate the shape of the function. 
 
The script for downloading the files with the halos containing satellite galaxies is given below:
\lstinputlisting{qn3.py}

